\documentclass[11pt]{article}


% Use wide margins, but not quite so wide as fullpage.sty
\marginparwidth 0.5in 
\oddsidemargin 0.25in 
\evensidemargin 0.25in 
\marginparsep 0.25in
\topmargin 0.25in 
\textwidth 6in \textheight 8 in
% That's about enough definitions

% multirow allows you to combine rows in columns
\usepackage{multirow}
% tabularx allows manual tweaking of column width
\usepackage{tabularx}
% longtable does better format for tables that span pages
\usepackage{longtable}
\usepackage[utf8]{inputenc}
\usepackage{color}
\usepackage{listings}
\usepackage{graphicx}
\usepackage{amsmath}
\definecolor{codegreen}{rgb}{0,0.6,0}
\definecolor{codegray}{rgb}{0.5,0.5,0.5}
\definecolor{codepurple}{rgb}{0.58,0,0.82}
\definecolor{backcolour}{rgb}{0.95,0.95,0.92}


\lstset{language=Matlab}



\begin{document}

\begin{titlepage}
	\vspace*{\stretch{1.0}}
	\begin{center}
		\Large\textbf{ASP Report: Section 1}\\
		\large\textit{Yiren (Aaron) Zhao}
	\end{center}
	\vspace*{\stretch{2.0}}
\end{titlepage}

\section{Statistical estimation}
\subsection{Sample mean}
To generate a uniform random variable $X\sim\mathcal{U}(0,\,1)$, we need to recall the \textbf{rand} function in Matlab, thus we obtain a vector of samples, denoted by $x[n] = [x[1],x[2]...x[1000]]^{T}$. In order to help us understand and uniform the representation of later calculations, we recall the probability density function of uniform distribution, denoted by $f(x)$, is defined as: \\
\begin{equation}
f(x) = \frac{1}{b-a} \quad \textbf{for} \quad a \leq x \leq b
\end{equation}  
For the generated samples, the value of $a$ and $b$ are respectively 0 and 1.\\
\\
We applied this function to generate 1000 samples in order to investigate the difference and relationship between theoretical mean and sample mean. We denote symbol $E(X)$ to represent the expected value of given random variable $X$, and $m$ is used to represent the theoretical mean. Therefore, we conclude an obvious relationship that is $m = E(X)$. We define that sample mean as $\hat{m}$, similarly, $\hat{m} = E(\hat{X})$. The uniform distribution is a continuous probability distribution, therefore, we define a theoretical mean for a continuous random variable $X$ as following:\\
\begin{equation}
m = E(X)=\int_{-\infty}^\infty xf(x) \mathrm{d}x
\end{equation}
We also define the sample mean equation as:\\
\begin{equation}
\hat{m} = E(\hat{X}) = \frac{1}{N} \sum_{n=1}^{N} x[n]
\end{equation}
The sampled mean is being treated as discrete data. Notice the summation starts from index 1, this is because in Matlab the first element of a vector corresponds to an index of 1. 



\section{Stochastic Process}
\section{Estimation of Probability distribution}



\end{document}          
